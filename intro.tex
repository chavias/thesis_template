%%% CHAPTER%%%

\chapter{Introduction to NMR Theory}
\label{chapter:intro}
\vspace{1cm}
%\textcolor{blue}{Might be better to remove all hats}
For the description of spin dynamics of NMR, classical and quantum mechanical theories are used.
The former was pioneered by Bloch and Torrey \cite{Bloch1946, Torrey1956}. This framework is quite simple and intuitive and treats the nuclear polarization as a magnetization vector without considering the quantum-mechanical nature of the nuclear spins.
A more precise description can be obtained using
% (non-relativistic)
quantum mechanics, which is better suited to describe the nature of nuclear spins subjected to a magnetic field.
In general, it is sufficient to treat the magnetic fields as classical and non-relativistic
vector fields, to describe the spin dynamics relevant for NMR. In the following, a short introduction to the basics of NMR theory for spin-$\frac{1}{2}$ nuclei using quantum mechanics and classical fields is presented.

\section{Interactions in Solid-State NMR}
All the described interactions discussed in this thesis share the same nature - they are ultimately interactions between magnetic fields.
However, magnetic fields can stem from different sources.
On one hand, we have the magnetic moments caused by the spins of the sample, and on the other hand, we have the external magnetic fields, which are not generated by the sample.
Therefore it is convenient to classify the relevant interactions for NMR spectroscopy into two groups: Interactions of a nuclear spin with an external magnetic field
\begin{equation}
\mathcal{{H}}^{(i,B)}={\vec{\textrm{I}}}_i\mathbf{A}^{(i,B)}\vec{B}~,
\label{eq:spin-field}
\end{equation}
and interactions between two spins $i$ and $j$
\begin{equation}
\mathcal{{H}}^{(i,j)}={\vec{\textrm{I}}}_i\mathbf{A}^{(i,j)}{\vec{\textrm{I}}}_j~,
\label{eq:spin-spin}
\end{equation}
Here $\vec{B}$ is the magnetic field vector of the external field, ${\vec{\textrm{I}}}$ is the vector containing the spin operators, and $\mathbf{A}$ is a tensor that describes the strength and angular dependency of the interactions. For spin-$\frac{1}{2}$ nuclei, the spin operators can be represented with scaled Pauli matrices:
\begin{equation}
{\textrm{I}}_x = \frac{1}{2}
\begin{pmatrix}
0 & 1 \\
1 & 0 
\end{pmatrix}
\textrm{,~~~~~~~~~~~~}{\textrm{I}}_y = \frac{1}{2}
\begin{pmatrix}
0 & -i  \\
i & 0  
\end{pmatrix}
\textrm{,~~~~~~~~~~~~}{\textrm{I}}_z = \frac{1}{2}
\begin{pmatrix}
1 & 0\\
0 & -1 
\end{pmatrix}
\textrm{.}
\label{eq:pauli}
\end{equation}
All Hamiltonians in this Chapter are given in units of angular frequency, where $\mathcal{{H}} = \frac{H}{\hbar}$.

\subsection{Zeeman Interaction}
The interaction of a nuclear spin with a static magnetic field $\vec{B}_0$ is called Zeeman interaction.
In a static field, the energy difference between the two energy levels of a single  spin-$\frac{1}{2}$ nucleus ${\vec{\textrm{I}}}_k$ is given by the Larmor frequency:
\begin{equation}
	\omega_0 = -\gamma_k \vec{B}_0~.
	\label{eq:larmor}
\end{equation}
By convention, the z-axis is chosen along $\vec{B}_0$ such that $\vec{B}_0 = (0,0,\textrm{B}_0)$. In these coordinates, the Zeeman Hamiltonian is expressed via the Larmor frequency and the z\nobreakdash-component of the spin operator ${\vec{\textrm{I}}}$ (Eq.~\ref{eq:pauli}):
\begin{equation}
\mathcal{{H}}_{\textrm{Z}} = \omega_0 {\textrm{I}}_{{z}}~.
\end{equation}
In conventional NMR, where typically strong static magnetic fields are utilized, the Zeeman interaction is the strongest interaction. 
However, it depends only on the strength of the magnetic field and the gyromagnetic ratio $\gamma_k$ of a nuclear isotope \textit{k} and therefore leads only to
a sparse amount of information by itself.
\subsection{Chemical Shift}
So far we only considered a single nucleus in a homogeneous and static magnetic field. However, the electrons in the sample experience the same static magnetic field. Due his will cause a local magnetic field $\vec B_{\ind}$ that adds to the static magnetic field $\vec B_0$
\begin{align}
  \vec B_{tot} = \vec B_0 + \vec B_{\ind}~.
\end{align}
In almost all materials is the linear dependence between the local field $\vec B_{\ind}$ and $\vec B_0$ an excellent approximation, and can be described by the magnetic susceptibility $\chi$ as
\begin{align}
  \vec B_{\ind} = \underline{\chi} \vec B_{0}~.
\end{align}
The magnetic susceptibility allows a simple classification of the material, according to the response to an external magnetic field: an alignment with the magnetic field,  $\chi > 0$, is called paramagnetism, or an alignment against the field, $\chi < 0$, is called diamagnetism. 
%\textcolor{blue}{Here something about diamagnetism etc could be written. Also about the macroscopic Chi and that it does not matter, Finally the BMS shift could be mentioned.}
%However
The part of the induced field that changes on the length scale smaller than molecules is described by the chemical shift tensor:
\begin{align}
	\vec{B}_{\ind}=\underline{\sigma}\vec{B}_0 = 
	\begin{pmatrix}
	\sigma_{{xx}} & \sigma_{{xy}} & \sigma_{{xz}} \\ 
	\sigma_{{yx}} & \sigma_{{yy}} & \sigma_{{yz}} \\ 
	\sigma_{{zx}} & \sigma_{{zy}} & \sigma_{{zz}} 
	\end{pmatrix}
	\begin{pmatrix}
	0 \\
	0 \\
	\textrm{B}_0
	\end{pmatrix}
	~.
\end{align}
As a consequence, the resonance frequency of that particular nucleus (Eq.~(\ref{eq:larmor})) is shifted slightly, depending on its electronic environment.
Following Eq.~(\ref{eq:spin-field}) leads to the definition of the chemical-shift Hamiltonian as  
\begin{equation}	\mathcal{{H}}_{\text{CS}}=\omega_0\left(\sigma_{xz}{\textrm{I}}_{x}+\sigma_{yz}{\textrm{I}}_{y}+\sigma_{zz}{\textrm{I}}_{z}\right)~.
	\label{eq:CS_normal}
\end{equation}
where, as before, the z-axis is chosen along the external magnetic field. In general, the chemical shift has an anisotropic part, called chemical-shift anisotropy (CSA), which dependents on the orientation of the molecules. In a powder sample, which contains a uniform distribution of orientations of the molecules the anisotropy results in heterogeneous line broadening as shown in Fig.~\ref{fig:CS-ernst}.\\
To characterize the heterogeneous broadening it is convenient to decompose the chemical shift into three components: The isotropic chemical shift
\begin{align}
  \sigma_{\textrm{iso}}  = \frac{Tr(\sigma)}{3}~,
\end{align}
its anisotropy
\begin{align}
  \delta=\bar \sigma_{zz}-  \sigma_{\textrm{iso}}~, 
\end{align}
and its asymmetry
\begin{align}
  \eta=\frac{\bar \sigma_{yy}-\bar\sigma_{xx}}{\delta}~,
\end{align}
where definition of the tensor elements follows the convention $|\bar\sigma_{zz}  - \sigma_\iso | \leq |\bar\sigma_{xx} -\sigma_{\iso} | \leq |\bar\sigma_{yy} - \sigma_{\iso}|$.
Here, $\bar \sigma$ is the chemical shift in the basis where the chemical shift is diagonal, called the principal axes system (PAS). 
In high static magnetic fields, the Zeeman interaction generally is orders of magnitude larger than the chemical\nobreakdash-shift interaction. In this case, it is useful to define an axis system that rotates around the z-axis at the Larmor frequency $\omega_0$. In this rotating frame, terms of Eq.~(\ref{eq:CS_normal}) that are proportional to ${\textrm{I}}_x$ and ${\textrm{I}}_y$ rotate around the z-axis with $\omega_0$, and are averaged. Typically, for high static magnetic fields, it can be assumed that the ${\textrm{I}}_x$ and ${\textrm{I}}_y$ components are averaged out completely. In this high\nobreakdash-field or \textit{secular} approximation, the chemical\nobreakdash-shift Hamiltonian simplifies to:
\begin{equation}
\mathcal{{H}}_{\text{CS}}\approx\omega_0\sigma_{zz}{\textrm{I}}_{z}~.
\label{eq:CS_secular}
\end{equation}
While NMR spectra principally involve resonance frequencies in Hertz, by convention, their axes report a chemical shift $\delta$ in parts\nobreakdash-per\nobreakdash-million (ppm), relative to a reference compound, and independent of the applied magnetic field $\vec B_0$. This is defined as:
\begin{equation}
\delta\left[\textrm{ppm}\right] = 10^6\cdot\frac{\nu_{\textrm{obs}}-\nu_{\textrm{ref}}}{\nu_{\textrm{ref}}}~.
\end{equation}
Here, $\nu_{\textrm{obs}}$ is the observed resonance frequency, $\nu_{\textrm{ref}}$ is the resonance frequency of the reference compound, and $\delta$ is the reported chemical shift.
\begin{figure}[H]
  \centering
  \includegraphics[width=0.5\textwidth]{CS-Ernst.png}
  \caption{\label{fig:CS-ernst} Simulated powder lineshape or the chemical shift anisotropy. Figure taken from Ernst et al. \cite{Script}}
\end{figure}
\subsection{Interaction with Time-Dependent Radio-Frequency Fields}
The interaction with time-dependent radio-frequency fields, denoted with $\vec B_1$, can be described similar to interactions with the static magnetic field.
As before it is sufficient to model the magnetic field as a classical vector field.
In general, this vector field will be spatial-dependent, not only because of the geometry of the coil generating it but also because of the magnetic susceptibility of the sample as discussed before. In Chapter \ref{chapter:optimization}, a general description with a spatial dependent. Often, the spatial dependence can be neglected, because the influence of the magnetic susceptibility (including the chemical shift)
is small. 
Usually, the $B_1$ field is generated by a single coil, perpendicular to the external magnetic field, which defines the x-axis of the laboratory coordinate system.
Under this approximations and conventions, the magnetic field has the form $\vec B_1(t) = B_{1x}(t)\vec e_x$.
Usually, the time-dependent $B_1$-field can be described as a wave with a constant frequency $\omega_{\rf}$, which is close to the Lamor frequency $\omega_0$. As a result the $B_1$-field takes the form
\begin{align}
  \vec B_1(t) = 2 B_1(t)\cos(\omega_\rf t + \phi(t)) \vec e_x=  2 B_1(t)\operatorname{Re}\{\exp(i (\omega_\rf t + \phi(t))\} \vec e_x~.
\end{align}
The amplitude $B_1(t)$ and phase $\phi(t)$ are, in general, time dependent, however on a
much longer time scale than $1/\omega_{\rf}$.
Following Eq. (\ref{eq:spin-field}) we obtain the radio-frequency Hamiltonian
\begin{align}
  \HH^{(k)}_{\rf}(t)=- 2 \gamma_k B_1(t)\cos(\omega_\rf t + \phi(t))\II_{kx}~.
\end{align}
This Hamiltonian can be simplified using an approximation, which is usually called the rotating wave approximation.
The basic idea of this approximation is that we express a wave as a superposition of a fast and slow oscillating component using $\cos (\omega_a t) \cos (\omega_b t) = \frac{1}{2}(\cos ((\omega_a -\omega_b)t) + \cos ((\omega_a +\omega_b)t))$ and neglect the fast oscillating part, because it averages out in the timescale of interest.
Transforming into a frame rotating with $\omega_{\rf}$ around z-direction and applying the rotating wave approximation leads to
\begin{align}
  \HH^{(k)}_{\rf}(t) &= \frac{\gamma_{k} B_1(t)}{2}\left[(e^{-i\phi(t)}+e^{i(2\omega_rt+\phi(t)})\II^{+}_{k} + h.c.\right]\\
 &\approx \frac{\gamma_{k} B_1(t)}{2}\left[e^{-i\phi(t)}\II^{+}_{k}+ h.c.\right]\\
 &= \omega_1(t)[\cos(\phi(t))\II_{kx}+\sin(\phi(t))\II_{ky}]~.
\end{align}
The neglected terms oscillating with $2\omega_\rf$ lead to a shift in the resonant line, called Bloch-Siegert shift \cite{Bloch:1940wy}. In most cases, these shifts are small and not important for the experiments presented in this thesis.
% The equilibrium magnetisation is described by a net magnetic moment of the nuclear spins in the sample, aligned along the z\nobreakdash-axis of the static magnetic field. At equilibrium, this magnetization is `in the shadow' of the external field, and to observe the nuclear spins, the net magnetic moment must be tilted into the x/y\nobreakdash-plane. In this orientation, the magnetic moment will precess around the z\nobreakdash-axis at the Larmor frequency (Equation~\ref{eq:larmor}). To achieve such a tilt, a  radio-frequency (rf) pulse is applied with the use of a small coil around the sample, which generates a field commonly referred to as the B$_1$ field. Since B$_1$ is a pulse-generated field it is time-dependent. Like the rotating frame defined for the chemical-shift interaction described above, an interaction frame can be defined that rotates around the z\nobreakdash-axis with $\omega_\textrm{rf}$, the frequency of the B$_1$ field. In this interaction frame, the B$_1$ field is time-independent, and in analogy to the Larmor frequency, a nutation frequency $\omega_1$ can be defined for a spin \textit{k} as:
% \begin{equation}
% 	\omega_1=-\gamma_k \textrm{B}_1~.
% \end{equation}
% A transformation to the interaction frame also removes time-dependencies of the x- and y\nobreakdash-components of magnetic moment precessing in the x/y\nobreakdash-plane. The radio-frequency Hamiltonian is then given by:
% \begin{align}
% \mathcal{{H}}_{\textrm{RF}} &= \omega_1{\vec{\textrm{I}}} \nonumber \\
%  &= \omega_1({\textrm{I}}_{x}\cos\psi+{\textrm{I}}_{y}\sin\psi)~.
% \end{align}
% where $\psi$ contains the phase of the rf-pulse.

% \noindent Assuming an ideal on-resonance pulse ($\omega_\textrm{rf}=\omega_0$) with phase $\psi = 0$, the resulting B$_1$ field generates a rotation of the magnetisation vector around the interaction-frame x\nobreakdash-axis with the nutation frequency $\omega_1$. The time $\tau$ and the nutation frequency $\omega_1$ determine the flip angle $\vartheta$. With the right $\tau$, a flip angle $\vartheta =$ 90$^{\circ}$ will bring the full magnetisation in the x/y\nobreakdash-plane and generate maximum signal intensity.

\subsection{Dipolar Coupling}
So far we only considered the interaction of the nuclear spin and external magnetic fields, i.e., a static magnetic field and the time-dependent radio-frequency field, which could be expressed in the form of Eq.~(\ref{eq:spin-field}).
However, more than one nuclear spin is usually present in the sample. The spins can interact with each other either by direct dipolar coupling through space or indirect interactions, mediated by the electrons of chemical bonds.
The through-space dipolar coupling between two spins $i$ and $j$ can be described by the Hamiltonian
% \begin{equation}
% \mathcal{{H}}_{\textrm{D}}^{\textrm{homo}} = -\frac{\mu_0}{4\pi} \frac{\gamma_k\gamma_l \hbar}{r_{ij}^3}   \left( \frac{3}{r^2_{ij}} (\vec {{\textrm{I}}}_{i}  \vec r_{ij})(\vec {{\textrm{I}}}_{j}  \vec r_{ij}) - (\vec {{\textrm{I}}}_{i} \vec{{\textrm{I}}}_{j})\right)
% \label{eq:dip_coup_homo}
% \end{equation}
\begin{equation}
\mathcal{{H}}_{\textrm{D}}= \frac{\mu_0}{4\pi} \frac{\gamma_k\gamma_l \hbar}{r_{ij}^3}   \left((\vec {{\textrm{I}}}_{i}\cdot \vec{{\textrm{I}}}_{j})-3\, (\vec {{\textrm{I}}}_{i} \cdot \vec n_{ij})(\vec {{\textrm{I}}}_{j} \cdot \vec n_{ij})\right)~,
\label{eq:dip_coup_homo}
\end{equation}
where $\vec{n}_{ij}$ is a unit vector in the direction joining the nuclei as shown in Fig.~\ref{fig:intro-magnetic-dipol}.
\begin{figure}[ht]
  \centering
  \includegraphics[width=0.5\textwidth]{magnetic-dipol.png}
  \caption{Schematic drawing of the interaction between two nuclear spins $I_i$ and $I_j$. The design of this drawing was strongly inspired by the figure of Ernst et al. \cite{Script}}
  \label{fig:intro-magnetic-dipol}
\end{figure}
The dipolar coupling Hamiltonian can rewritten according to Eq.~(\ref{eq:spin-spin}), as 
\begin{align}
  \label{eq:dipolar-coupling-Hamiltonian}
  \HH_D = \II_i \mathbf{D}^{(ij)} \II_j~,
\end{align}
where $D_{\alpha\beta}^{(ij)} =  \frac{\mu_0}{4\pi} \frac{\gamma_k\gamma_l \hbar}{r_{ij}^3}(\delta_{\alpha\beta}- 3\, n_{ij,\alpha} n_{ij,\beta})$ is a traceless symmetric tensor and $\delta_{ij}$ the Kroneker symbol. The matrix $\mathbf{D}^{(ij)}$ is diagonal in the PAS:
\begin{align}
  \mathbf{D}^{(ij)}
  = -2 \frac{\mu_0}{4\pi}\frac{\gamma_i\gamma_j}{r_{ij}^3}
  \left(
  \begin{matrix}
    -\frac{1}{2} & 0 & 0\\
    0 & -\frac{1}{2} & 0\\
    0 & 0 & 1\\
  \end{matrix}
  \right)~.
\end{align}
% between the magnetic dipole moments of two spins that interact through space.
% The dipolar coupling Hamiltonian between two spins \textit{i} and \textit{j} is often defined in terms of the ``dipolar alphabet''
% \begin{equation}
% 	\mathcal{{H}}_{\textrm{D}} = \frac{\mu_0}{4\pi} \frac{\gamma_k\gamma_l \hbar}{r_{kl}^3}   \left(A+B+C+D+E+F\right)~,
% 	\label{eq:dip.coupl}
% \end{equation}
% where $\mu_0$ is the vacuum permeability, $\gamma_k$ and $\gamma_l$ are the gyromagnetic ratios of the two nuclei $k$ and $l$ involved in the interaction, $\hbar$ is the reduced Planck constant and $r_{kl}$ is the internuclear distance between spins $k$ and $l$. The single terms $A$ to $F$ are given by
% \begin{align}
% A &= 2 {\textrm{I}}_{kz} {\textrm{I}}_{lz} \frac{(3\cos^2\vartheta-1)}{2} \nonumber \\
% B &= -\frac{1}{2} ({\textrm{I}}_{k}^{\textrm{+}} {\textrm{I}}_{l}^{\textrm{-}}  + {\textrm{I}}_{k}^{\textrm{-}} {\textrm{I}}_{l}^{\textrm{+}} )\frac{(3\cos^2\vartheta-1)}{2} \nonumber \\
% C &= ({\textrm{I}}_{k}^{\textrm{+}} {\textrm{I}}^{\textcolor{white}{.}}_{lz}  + {\textrm{I}}^{\textcolor{white}{.}}_{kz} {\textrm{I}}_{l}^{\textrm{+}} )\frac{3\sin\vartheta\cos\vartheta e^{-i\varphi}}{2} \nonumber \\
% D &= ({\textrm{I}}_{k}^{\textrm{-}} {\textrm{I}}^{\textcolor{white}{.}}_{lz}  + {\textrm{I}}^{\textcolor{white}{.}}_{kz} {\textrm{I}}_{l}^{\textrm{-}} )\frac{3\sin\vartheta\cos\vartheta e^{i\varphi}}{2} \nonumber \\
% E &= \frac{1}{2}{\textrm{I}}_{k}^{\textrm{+}} {\textrm{I}}_{l}^{\textrm{+}} \frac{3\sin^2\vartheta e^{-2i\varphi}}{2} \nonumber \\
% F &= \frac{1}{2}{\textrm{I}}_{k}^{\textrm{-}} {\textrm{I}}_{l}^{\textrm{-}} \frac{3\sin^2\vartheta e^{2i\varphi}}{2}~, \nonumber 
% \end{align}
% where ${\textrm{I}}^{\textrm{+}}={\textrm{I}}_{x}+i{\textrm{I}}_{y}$ and ${\textrm{I}}^{\textrm{-}}={\textrm{I}}_{{x}}-i{\textrm{I}}_{{y}}$; and the polar angles $\vartheta$ and $\varphi$ describe the orientation of the internuclear vector in the laboratory frame. The terms C, D, E, and F contain transverse spin-operator components (regarding ${\textrm{I}}_x$ or ${\textrm{I}}_z$) that are time-dependent in an interaction frame that rotates around the z\nobreakdash-axis. These terms can be neglected in the secular approximation, and for two spins $k$ and $l$ of the same isotope, the homonuclear dipolar coupling Hamiltonian reduces to:
In a rotating frame, which is rotating around the axis of the static magnetic field ($\II_{z}$) with the Larmor frequency, the Hamiltonian of a homonuclear coupled spin pair is simplified further under the secular (high-field) approximation, where only terms are kept which are not averaged out due to the strong Zeeman interaction
\begin{equation}
\mathcal{{H}}_{\textrm{D}}^{\textrm{homo}} = -\frac{\mu_0}{4\pi} \frac{\gamma_i\gamma_j \hbar}{r_{ij}^3}\frac{(3\cos^2\vartheta-1)}{2}   \left(2\,{\textrm{I}}_{iz} {\textrm{I}}_{jz}  - \frac{1}{2}({\textrm{I}}_i^+ {\textrm{I}}_j^- + {\textrm{I}}_i^- {\textrm{I}}_j^+)\right)~.
\label{eq:dip_coup_homo}
\end{equation}
Analogously, we can simplify the Hamiltonian of a heteronuclear spin pair, using a  (double) rotating frame, rotating around the axis of the static magnetic field ($\II_z$ and $\SSS_z$) with the corresponding Lamor frequencies:
\begin{align}
\mathcal{{H}}_{\textrm{D}}^{\textrm{hetero}} =- \frac{\mu_0}{4\pi} \frac{\gamma_i\gamma_j \hbar}{r_{ij}^3}\frac{(3\cos^2\vartheta-1)}{2}    2\,{\textrm{I}}_{iz} {\textrm{S}}_{jz}~.  
\end{align}
% \begin{equation}
% \mathcal{{H}}_{\textrm{D}}^{\textrm{hetero}} = \frac{\mu_0}{4\pi} \frac{\gamma_k\gamma_l \hbar}{r^3_{kl}}\frac{(3\cos^2\vartheta-1)}{2}  2{\textrm{I}}_{k{z}} {\textrm{S}}_{l{z}}~.
% \end{equation}
\subsection{J Coupling}
As mentioned, the nuclear spins can interact directly through space via the dipolar interaction, or indirectly through the electron spins of the chemical bond, known as J-coupling.
In accordance with Eq.~(\ref{eq:spin-spin}), the J-coupling Hamiltonian is given as:
\begin{equation}
\mathcal{{H}}_{\textrm{J}} = 2 \pi {\vec{\textrm{I}}}_k \mathbf{J}^{(k,l)} {\vec{\textrm{I}}}_l~.
\end{equation}
The J-coupling contains anisotropic contributions described by the tensor $\mathbf{J}^{(k,l)}$. Most of the time, the anisotropic contributions are small and impossible to distinguish from the dipolar coupling. For this reason, usually only the isotropic part of $\mathbf{J}^{(k,l)}$ is considered. In contrast to the dipolar coupling, the isotropic value of the J-coupling is not zero. 
%The isotropic J-coupling Hamiltonian in the laboratory frame is then:
%\begin{equation}
%\mathcal{{H}}_{\textrm{J}} = 2 \pi J_{\textrm{iso}}^{(k,l)} {\vec{\textrm{I}}}_k \cdot {\vec{\textrm{I}}}_l~.
%\end{equation}
Again, a homonuclear and heteronuclear case can be separately defined in secular approximation as:
\begin{align}
\mathcal{{H}}_{\textrm{J}}^{\textrm{homo}} &= 2 \pi J_{\textrm{iso}}^{(k,l)} {\vec{\textrm{I}}}_k \cdot {\vec{\textrm{I}}}_l ~,\\
\mathcal{{H}}_{\textrm{J}}^{\textrm{hetero}} &= 2 \pi J_{\textrm{iso}}^{(k,l)} {\textrm{I}}_{kz} {\textrm{S}}_{lz}~.
\end{align}

\section{Density Operator}
\label{sec:density-operator}
The NMR sample typically consists of a large ensemble of $\sim10^{22}$ individual NMR-active nuclear spins. This spin ensemble is commonly in thermal equilibrium at the beginning of an NMR experiment, where the spins inhabit many different (superposition) states. 
Therefore it is usually more convenient to consider the whole ensemble using a density operator than the wave function of each individual spin.
%Here a short overview of the formalism for usage in magnetic resonance is given
A detailed presentation of the density operator formalism can be found in literature \cite{nielsen_chuang_2010}.
Here a short overview of the formalism for usage in NMR is given.
%, which shows how all postulates of quantum mechanics can be reformulated using the density operator and when this approach becomes useful.
The density operator for a system is defined by the equation:
\begin{align}
  \rho = \sum_j p_j\ket{\psi_j}\bra{\psi_j}~,
\end{align}
with $Tr(\rho)=\sum_j p_j=1$.
% In a closed quantum system, as we assume in NMR (NOT true), the time evolution of the density operator mediated by a unitary operator, hence
% \begin{align}
%   \UU \rho \UU^\dagger = \sum_j p_j \UU \ket{\psi_j}\bra{\psi_j} \UU^\dagger
% \end{align}
The time evolution of the density operator is given by the Liouville-von Neumann equation
\begin{align}
  \label{eq:Liouville-von Neumann}
  i\hbar \, \dot{\rho}(t) = [\HH,\rho(t)]~.
\end{align}
As mentioned before the initial density operator in magnetic resonance is normally in thermal equilibrium and only subjected to the external magnetic field and therefore can be described as a canonical ensemble: 
\begin{align}
\label{eq:rho0-complete}
 \rho_0=\frac{e^{-\beta {\hbar \HH_\mathrm{Z}}}}{Tr(e^{-\beta {\hbar \HH_\mathrm{Z}}})}~.%=\frac {1}{Z(\beta )} \sum_{n}\ket{\psi_{n}} e^{-\beta E_{n}}\bra{\psi_{n}}
\end{align}
with $\beta = \frac{1}{k_B T}$ and where $\HH_\mathrm{Z}= -\omega_0 \II_z$ is the Zeeman Hamiltonian. Note, that $\hbar$ appears in the equation because $\HH_\mathrm{Z}$ is given in angular frequency.
For modest fields and typical nuclei at room temperature the high-temperature approximation $(\hbar\gamma B \ll \beta^{-1}\approx 10^{4})$ is appropriated, which simplifies Eq.~(\ref{eq:rho0-complete}) to 
\begin{align}
  \rho_0 = 2^{-n}[\mathbf{1}-\beta \,\hbar\, \HH_\mathrm{Z}]~,
\end{align}
for a $n$ spin system.

\section{Spherical Tensor Representation}
\label{sec:spherical tensor}
%\textcolor{blue}{maybe it wou ld be nice to say why we want to understand rotations of the Hamiltonian, powder sample, irradiation etc.}
% Rotations of the Hamiltonian often necessary in solid state NMR.
% For example to describe the powder sample consisting of many molecules with different orientation
% we have to apply rotations\\
%In order to obtain describe rotations of the Hamiltonian i

In magnetic resonance, it is advantageous to formulate the Hamiltonian in a symmetry-adapted way, i.e, separate the Hamiltonian according to the transformation behavior under a general rotation. 
Therefore, spherical tensors are useful, which are defined by their transformation rule under a general rotation $\mathcal{D}$:
\begin{align}
   \mathcal{D}^{\dagger}  \vec{A}^{(i)}_{l,m}  \mathcal{D} = \sum^l_{m'=-l} \mathcal{D}^{l}_{m,m'}\vec{A}_{l,m'}~.
\end{align}
The appeal of writing the Hamiltonian in terms of spherical tensors becomes evident; the transformation rule ensures that the rotation of a spherical tensor preserves the value of $l$, known as rank, so that the rotation is block-diagonal, in fact irreducible, in terms of the spherical tensor decomposition.
The matrix $\mathcal{D}^{l}_{m,m'}= \bra{lm'} \mathcal{D}\ket{lm}$ is the Wigner D-matrix,
which is defined in the spherical basis $\ket{lm}$ given by
\begin{align}
  {\vec{\II}\,}^2 \ket{lm} &= l(l+1) \ket{lm}~,\\
  \II_z \ket{lm} &= m \ket{lm}~.
\end{align}
For real spatial rotations, i.e. elements of SO(3), $l = 0, 1, 2, ...$ , whereas for SU(2), $l = 0, 1/2, 1, 3/2, ...$ and $m = -l,-l+1 , ..., l$ for both cases.
In three dimensions a general rotation can always be written as three successive rotations:
\begin{align}
  \mathcal {D}(\alpha ,\beta ,\gamma )=e^{-i\alpha \II_{z}}e^{-i\beta \II_{y}}e^{-i\gamma \II_{z}}~,
\end{align}
where $\alpha, \beta, \gamma$ are known as Euler angles.
As a result the Wigner $\mathcal{D}$-matrix becomes:
\begin{align}
  \mathcal{D}^{l}_{m',m}(\alpha,\beta,\gamma) =
  \bra{lm'} \mathcal{D}(\alpha ,\beta ,\gamma )\ket{lm} =e^{-im'\alpha }d_{m'm}^{l}(\beta)e^{-im\gamma }~,
\end{align}
with
\begin{align}
 d_{m'm}^{l}(\beta )=\langle lm'|e^{-i\beta J_{y}}|lm\rangle = \mathcal{D}_{m'm}^{l}(0,\beta ,0)~.
\end{align}
Because rotations of the spatial and the spin part are independent of each other, the Hamiltonian can be formulated  as a scalar product between two spherical tensors $A$ and $\mathcal{I}$, which encode the spatial part and the spin part of the Hamiltonian respectively:
\begin{align}
\HH =   \sum_i \sum^2_{l=-2} \vec{A}^{(i)}_l \vec{\mathcal{I}}^{(i)}_l = \sum_i \sum^{2}_{l=-2}\sum_{q=-l}^l (-1)^q A^{(i)}_{l,q} \mathcal{I}^{(i)}_{l,-q}~.
\end{align}
Here $i$ is the index of the interactions, $l$ the rank and $q$ the index of the sub-component.
Again, rotation of spherical tensors preserves the rank and therefore provides a symmetry-adapted representation of the Hamiltonian.
%\textcolor{blue}{Spin tensor is product of two first rank tensors}


\subsection{The Rotating Frame}
\label{sec:secular}
As we have discussed before, the Zeemann interaction causes a uniform rotation of the spin part of the Hamiltonian around the static magnetic field along the z-axis with the Lamor frequency $\omega_0$, which can be expressed by the Wigner $\mathcal{D}$-matrix
$\mathcal{D}_{m'm}^{l}(- \omega_0 t, 0 ,0) = e^{i m' \omega_0 t}d_{m'm}^{l}(0)$. In the high field or secular representation, where $\omega_0$ is large, the rotation averages out all components with $m'\neq 0$, i.e. all spin components that are not along the z-axis. 
As a result the Hamiltonian in the secular approximation has the form 
\begin{align}
  \HH =  \sum_i \sum^{2}_{l=-2} A^{(i)}_{l,0} \mathcal{I}^{(i)}_{l,0}~.
\end{align}
\section{Magic-Angle Spinning}
Anisotropic interaction provides a large amount of information since their tensor components manifest the internal structure and dynamic of the molecule. In a powder sample, it manifests as a specific broadening (powder pattern) of a line due to the different orientations of the molecule in the sample. In a sample with many spins the broad lines often overlap, covering its features and strongly limiting the information that can be extracted.
%\textcolor{blue}{In a liquid, the anisotropic interactions are averaged out by tumbling of the molecule around its center of gravity}
A common approach to this problem is to reduce the broadening of the lines by spatial rotation of the sample.
Again we can use the spherical tensor notation, as described in Section \ref{sec:spherical tensor} to express a three-dimensional rotation using the Euler angles $\alpha, \beta, \gamma$ with the Wigner $\mathcal{D}$-matrix elements as 
\begin{align}
  A_{l m'} &= \sum_{m=-l}^{l}\mathcal{D}^{l}_{m,m'}(-\omega_{\rr} t,-\theta_{\rr},0) A^{\text{rot}}_{lm}~,
%          &= e^{-im'\alpha }d_{m'm}^{l}(\beta)
\end{align}
where $\omega_{\rr}$ is the rotation frequency of the sample, and $\theta_{\rr}$ the angle between the rotation axis of the sample and the static magnetic field. $A^{\text{rot}}_{lm}$ is the spatial part of the Hamiltonian in a rotor-fixed frame, where the z-axis is along the rotor. The secular approximation ($m'= 0$) leads to
\begin{align}
    A_{l 0} &=\sum_{m=-l}^{l} e^{-i m \omega_{\rr} t }d_{m 0}^{l}(-\theta_{\rr})A^{\text{rot}}_{lm}~.
\end{align}
If the free induction decay (FID) is sampled at multiples of the rotation frequency $m = \pm 1$ and $m = \pm 2$ vanishes, otherwise it will lead to spinning sidebands, which decrease in intensity with higher spinning frequency. Therefore $m'=0$ leads to the most important contributions proportional to $d_{m 0}^{l}(\theta_{\rr})$ depicted in Figure \ref{fig:d-spinning}  as a function $\theta_{\rr}$. Note, that every rank can be averaged out at a different angle $\theta_{\rr}$. The second-rank component proportional to
$
  d^2_{00}= \frac{3cos^2(\theta_{\rr})-1}{2}
$
is often the dominant contribution to the NMR linewidth, especially for spin-$1/2$ system. Therefore the angle, where the second-rank contributions average out, is especially important and therefore called the magic angle $\theta_{\mathrm{m}}$ defined by $\theta_\mathrm{m}=\arccos\left(1/\sqrt 3 \right)\approx 54.74^\circ$.
\begin{figure}[h]
  \centering
  \includegraphics[width=0.8\textwidth]{rotationMAS.png}
  \caption{The angular dependence of the reduced Wigner rotation matrix elements
 shows the differences in scaling by rotation about a single at an angle with the direction of the static magnetic field. Figure taken from Ernst et al. \cite{Script}}
  \label{fig:d-spinning}
\end{figure}

\newpage

\section{Unitary Transformations}
\label{sec:unitary-trafo}
In principle, it is possible to calculate the dynamics of the spin system, once the Hamiltonian is known.
However, solving the Liouville-von Neumann equation (Eq.~(\ref{eq:Liouville-von Neumann})) can be difficult, even using a computer.
% \textcolor{blue}{More explanations would be nice ...}
One method to simplify the problem and gain more physical insight is to change to a more suitable frame of reference.
Which frame is more suitable depends on the problem at hand, i.e what physical insight we pursue or if solely a simplification of the problem is desired. The different choices for the interaction frame for solid-state NMR and the advantages and disadvantages of the different frames are discussed in detail in Chapter \ref{chapter:chemical-shift-interaction-frame}.
\\
In the following, we express the dynamics of the density operator in the new frame with the Hamiltonian $\HH$ and density operator $\rho$ of the initial frame together with the unitary operator which mediates the change of frame. Therefore we insert $\UU\UU^{\dagger} = \mathbf{1}$ in the Liouville von-Neumann equation (Eq.~(\ref{eq:Liouville-von Neumann})):
\begin{align}
  &\hphantom{\,\implies\,\,}\dot{\rho} = \UU\UU^{\dagger} \dot{\rho}\UU\UU^{\dagger} = \UU \left( \partial_t (\UU^{\dagger}\rho \UU) - \dot{\UU}^{\dagger} \rho \UU - \UU^{\dagger} \rho \dot{\UU}\right)\UU^\dagger = -i [H,\rho]~,
\end{align}
where we used the chain rule of the derivative.
Rearranging the equation and using $\partial_t{\UU^{\dagger}\UU}=\dot{\UU}^\dagger\UU+\UU^\dagger\dot{\UU}$ leads to
\begin{align}
  \UU\partial_t (\UU^{\dagger}\rho \UU)\UU^\dagger &= -i [H,\rho] - \UU\dot{\UU}^\dagger\rho \UU\UU^\dagger-\UU\UU^\dagger\rho \dot{\UU}\UU^\dagger \\
  &= -i [H,\rho] - [\dot{\UU}\UU^\dagger,\rho]~.
\end{align}
Now we just have to multiply $\UU^{\dagger}$ from the left and $\UU$ from the right to obtain
\begin{align}
  \partial_t (\UU^{\dagger}\rho \UU)&= -i\, \UU^{\dagger}[ \HH  + i \dot{\UU}\UU^\dagger, \rho]\UU\\
                                    &= -i\left(\,\UU^{\dagger}\HH\rho\UU-\UU^{\dagger}\rho\HH\UU+i\dot{\UU}\UU^\dagger(\UU^{\dagger}\rho\UU)-i(\UU^{\dagger}\rho\UU)\dot{\UU}\UU^\dagger\right)\\
                                         &=  -i[\UU^\dagger \HH \UU  + i \dot{\UU}^\dagger \UU, \UU^{\dagger}\rho \UU]~,
\end{align}
where again we made use of $\UU\UU^{\dagger} = \mathbf{1}$.
As a result, we obtain
\begin{align}
 \dot{\rho}'
  = -i\, [\HH', \rho']~,
\end{align}
where
\begin{align}
  \rho'&=\UU\rho\UU^\dagger~,\\
  \HH' &= \UU^\dagger \HH \UU  + i \dot{\UU}^\dagger \UU~.
\end{align}
The latter two equations describe the transformation of the Hamiltonian and the density operator from an initial frame to a new frame, labeled with a dash, mediated by a unitary operator.
The concept of unitary transformation will be used extensively throughout this thesis.
%\section{Rotations of the spin space}
\subsection{Time-Independent Unitary Transformations: Tilted Frames}
The previous equations are valid for any unitary transformation, i.e. time-dependent and time-independent transformations. 
A time-independent unitary transformation ($\dot{\UU}=0$) is used to align a specific spin component with a specific axis resulting in a tilt of the initial frame, therefore often called tilted frame transformation.
As an example, consider the simple Hamiltonian
\begin{align}
  \HH(t) = \omega(t) 2\, \II_z\SSS_z +\omega_1(t) \II_x~.
\end{align}
Here it is convenient to align the irradiation with the z-axis, to arrive to more compact equations.
Therefore we use the time-independent unitary transformation
\begin{align}
  \UU = \exp\left(-i \frac{\pi}{2}\II_y\right)~,
\end{align}
which aligns the transforms the $\II_x$ to the $\II_z$ operator.
As a result, we obtain the Hamiltonian in the new frame
\begin{align}
  \HH' = \UU^\dagger \HH \UU = - \omega(t) 2\, \II_x\SSS_z +\omega_1(t) \II_z~.
\end{align}
If we further consider a density operator in thermal equilibrium as described in Section \ref{sec:density-operator}, we obtain
\begin{align}
  \rho' = 2^{-n}[\mathbf{1}-\beta \,\hbar\,  \omega_0 \II_x]~.
\end{align}
\subsection{Time-Dependent Unitary Transformations: Interaction Frames}
\label{sec:inter}
Let us now consider the case, where $\UU$ is time dependent and express the unitary operator using a time-dependent hermitian operator $\HH_I(t)$ as
\begin{align}
\label{eq:time-dependent-U}
\UU = \mathcal{T} \exp\left({-i \int \HH_I(t) dt}\right)~,
\end{align}
with the time-ordered product of operators 
\begin{align}
  \label{eq:time-ordering-operator}
  \mathcal {T}
  \left\{A(x)B(y)\right\}
  :=\begin{cases}A(x)B(y)
       &{\text{if }}\tau _{x}>\tau _{y}\\
       \pm B(y)A(x)&{\text{if }}\tau _{x}<\tau _{y}~.
     \end{cases}
\end{align}
where $\pm$ depends if the operators are bosonic or fermionic.
By convention, for bosonic operators the $+$ sign is chosen, whereas for fermionic particles, the sign depends on the number of operator interchanges necessary to achieve the proper time ordering. Here we are interested in spin-$1/2$ nuclei, therefore fermionic particles.  
As a direct consequence of Eq.~(\ref{eq:time-dependent-U}), we obtain $i \dot{\UU}\UU^{\dagger}= -\HH_I(t)$ and therefore
\begin{align}
  \label{eq:interaction-expression}
\HH' &= \UU^\dagger \HH \UU  - \HH_I~.
\end{align}
Let us consider the same example as in the previous Section, where we applied a tilted frame transformation, that resulted in 
\begin{align}
  \HH = - \omega(t) 2 \II_x\SSS_z +\omega_1(t) \II_z~.
\end{align}
We removed the dash on the Hamiltonian in comparison to the previous Section, because we now consider the tilted frame as the initial frame.
To simplify this Hamiltonian further, we can go into a frame, which incorporates the dynamic produced by $\omega_1(t) \II_z$, i.e.
\begin{align}
  \label{eq:time-dependent-U-with-H}
\UU = \mathcal{T} \exp\left({-i \int \omega_1(t') \II_z dt'}\right)~.
\end{align}
Following Eq.~(\ref{eq:interaction-expression}), we obtain for the transformed Hamiltonian
\begin{align}
  \HH' = - \omega(t)\UU^{\dagger}\II_x\UU\SSS_z = - \omega(t)
  \left(\II^+e^{i\int\omega_1(t')dt'}+ \II^-e^{-i\int\omega_1(t')dt'}\right)\SSS_z~.
\end{align}
Note, that the term incorporated in the interaction frame, i.e. $\HH_I$, is not explicitly present in the Hamiltonian anymore, because it is incorporated into the interaction frame.
The density operator in the new frame is
\begin{align}
  \rho' = 2^{-n}\left[\mathbf{1}-\,  \frac{\beta \hbar \omega_0}{2}
  \left(\II^+e^{i\int\omega_1(t)dt}+ \II^-e^{-i\int\omega_1(t)dt}\right) \right]~.
\end{align}
It is important to realize, that this density operator is time dependent, because of the interaction frame transformation.  However, the density operator $\rho'$ does \textit{only} include $\HH_I=\omega_1(t) \II_z$ and \textit{not} the complete Hamiltonian. To obtain the dynamic of the system the Liouville von-Neumann equation
$
 \dot{\rho}'
  = -i\, [\HH', \rho']
$
has still to be solved.
%%%%%%%%%%%%%%%%%%%%%%%%%%%%%%%%%%%%%%%%%%%%%%%%%%%%%%%%%%%%%%%%%%%%%%% 
%
%                  Derivation of the standard Floquet Theory
%
%%%%%%%%%%%%%%%%%%%%%%%%%%%%%%%%%%%%%%%%%%%%%%%%%%%%%%%%%%%%%%%%%%%%%%%
\section{Derivation of Floquet Theory for NMR}
\enlargethispage{1.5cm}
In this Section the derivation of Floquet theory for NMR is presented. Additional information to the derivation can be found  in literature \cite{Shirley:1965tn,Leskes:2010dx,Ivanov:2021uy}. 
We begin the derivation from differential equation in the propagator, which is a direct consequence of the Schr\"{o}dinger equation:
\begin{align}
\dot{\UU}(t) = -i\,\HH(t) \UU(t)~.
\label{eq:SG}
\end{align}
Floquet theory for magnetic resonance is based on the assumption, that the Hamiltonian is periodic and, therefore, the Schr\"{o}dinger equation is a periodic differential equation.
We can incorporate the periodicity of the Hamiltonian explicitly by expanding it in a Fourier series as
\begin{align}
\HH(t) = \sum_k \HH^{(k)}e^{i k \omega t}~.
\label{eq:H_FS}
\end{align}
The form of the solution of the periodic differential Eq. (\ref{eq:SG}) is given by the Floquet theorem \cite{Floquet:1883vq} as
\begin{align}
\UU(t) = u(t)e^{i \Lambda t}u^{-1}(0)~,
\label{eq:U_Floquet}
\end{align}
where $\Lambda$ is a diagonal real matrix because $\HH(t)$ is a Hermitian operator. The matrix $u(t)$ has the same periodicity as the Hamiltonian and can also be expressed as a Fourier series
\begin{align}
u(t) = \sum_{n} u_n e^{i n\omega t}~.
\label{eq:u_FS}
\end{align}
Note, that now the time dependence of the operators is only in the exponent and, therefore, inserting Eqs. (\ref{eq:H_FS}) - (\ref{eq:u_FS}) into Eq. (\ref{eq:SG}) will lead to the algebraic equation
\begin{align}
\sum_{k} (\HH^{(n-k)} +\omega \delta_{n k}) u_k  = \Lambda u_n~.
\label{eq:eigenvalue_equation_u}
\end{align}
This equation is equivalent to a time-independent Schr\"{o}dinger equation for the Fourier coefficients $u_n$, where the Hamiltonian
\begin{align}
\HH_F = \sum_{k} \HH^{(n-k)} +\omega \delta_{n k}~,
\end{align}
is called the Floquet Hamiltonian. \enlargethispage{1.cm}
It is convenient to represent the Floquet Hamiltonian and $u_n$ in a product space  of the spin Hilbert space and the Fourier space, where each basis state corresponds to a Fourier harmonic. Typically the basis states are written as $\ket{n,\mu}=\ket{n}\otimes\ket{\mu}$, where $\ket{\mu}$ denote the basis states of the spin Hilbert space, and $\ket{n}$ are the basis states of the Fourier space. Throughout this article Greek letters are used for the spin basis and Latin letters for the Fourier harmonics. Note, that there are infinitely many Fourier harmonics ($n\in \mathbb{Z}$), hence the matrix representation of the operators has infinite dimensions as well. The explicit representation of $\HH_F$ \cite{Scholz:2010hq,Leskes:2010dx,Ivanov:2021uy} is given by:
\begin{align}
\HH_F= \sum_{n} F_n\otimes\HH^{(n)} +\omega F_z\otimes \mathbf{1}~.
\label{eq:floquet_hamiltonian}
\end{align}
Here, the $F$ operators act on the Fourier space and are defined by $F_z\ket{n}=n\ket{n}$ and $F_n\ket{m}=\ket{m+n}$. In the Fourier space the Floquet Hamiltonian can be represented as a infinite matrix as shown in Figure \ref{fig:floquet-matrix}. %The higher the Fourier component of the Hamiltonian the further away it from the diagonal it appears in the Floquet matrix.  
The solution of  Eqs. (\ref{eq:eigenvalue_equation_u}) or (\ref{eq:floquet_hamiltonian}) leads to the formal solution of Eq. (\ref{eq:SG}) expressed in the spin Hilbert space:
\begin{align}
\UU(t) =  \sum_{n} \bra{n}\exp\left(i \HH_F t\right)\ket{0} e^{i n \omega t}~.
\end{align}
In conclusion, we solved the Schr\"{o}dinger equation (Eq. (\ref{eq:SG})) by  converting it to an algebraic equation using Fourier series expansions and the Floquet theorem assuming a periodic time-dependent Hamiltonian. As a basis we chose the product of the spin Hilbert space and Fourier space basis, which uses the Fourier harmonics.
\begin{figure}[H]
  \centering
  \includegraphics[width=0.5\textwidth]{Floquet-matrix.png}
  \caption{\label{fig:floquet-matrix} General matrix representation of the Floquet Hamiltonian (Eq.~(\ref{eq:floquet_hamiltonian})). Figure taken from Ivanov et al. \cite{Ivanov:2021uy}.}
\end{figure}

\subsection{Van Vleck Perturbation Theory}
\enlargethispage{0.5cm}
Typically van Vleck perturbation theory is applied on the Floquet Hamiltonian to obtain effective Hamiltonians. The procedure can be found in the literature \cite{Ramesh:2001uz,Vinogradov:2001ui,Ernst:2005ic,Scholz:2007bs, Leskes:2010dx, Ivanov:2021uy} and leads to the first, second and third-order Hamiltonian
\begin{align}
\bar{\HH}^{(1)} = \HH^{(0)}~,
\end{align}
\begin{align}
&\bar{\HH}^{(2)} = \frac{1}{2}
\sum\limits_{n\neq 0}
\frac{[\HH^{(n)}
	,\HH^{(-n)}]}
{n \omega }
\end{align}
and
\begin{align}
\begin{split}
  {\bar{{\HH}}}^{\left( 3 \right)} =& \frac{1}{2}\sum\limits_{n \ne 0} {\frac{\left[ \left[ {{\HH}}^{(n)},{{\HH}}^{(0)} \right],{{\HH}}^{( - n)} \right]}{{\left( {n{\omega _{\mathrm{r}}}} \right)}^2}}  + \frac{1}{3}\sum\limits_{k,n \ne 0} {\frac{\left[ {{\HH}}^{(n)},\left[ {{\HH}}^{(k)},{{\HH}}^{( - k - n)} \right] \right]}{{n{\omega _{\mathrm{r}}}k{\omega _{\mathrm{r}}}}}}~.
\end{split}
  \end{align}
Extending operator-based Floquet theory to multiple frequencies is straightforward. The details can be found in several reviews \cite{Scholz:2010hq,Leskes:2010dx,Ivanov:2021uy}.
% \section{stuff}
% The time ordered matrix exponential is defined as
% \begin{align}
% \mathcal {T}\exp \left(\int _{0}^{t}d\tau A(\tau )\right):&=\sum _{n=0}^{\infty }{\frac {1}{n!}}\iint \!\cdots \!\int _{0}^{t}d^{n}\tau \ {\mathcal {T}}\left[A(\tau _{1})A(\tau _{2})\cdots A(\tau _{n})\right]\\&=\sum _{n=0}^{\infty }\int _{0}^{t}d\tau _{1}\int _{0}^{\tau _{1}}d\tau _{2}\cdots \int _{0}^{\tau _{n-1}}d\tau _{n}\ A(\tau _{1})A(\tau _{2})\cdots A(\tau _{n})
% \end{align}
%%% Local Variables:
%%% mode: latex
%%% TeX-master: "thesis-main"
%%% TeX-master: "thesis-main"
%%% TeX-master: t
%%% TeX-master: "thesis-main"
%%% TeX-master: "thesis-main"
%%% TeX-master: "thesis-main"
%%% TeX-master: "thesis-main"
%%% TeX-master: "thesis-main"
%%% TeX-master: "thesis-main"
%%% TeX-master: "thesis-main"
%%% TeX-master: "thesis-main"
%%% End:
